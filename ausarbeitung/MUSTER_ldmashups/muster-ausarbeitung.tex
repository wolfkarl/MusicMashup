% Muster für die Seminarausarbeitung
% HPI Potsdam

\documentclass[11pt, a4paper]{article}

\usepackage{ngerman}
\usepackage[utf8]{inputenc} %Korrekte Kodierung der Umlaute nach UTF-8
\usepackage[T1]{fontenc} %Korrekte Kodierung der Umlaute nach UTF-8
\usepackage{amsfonts}
\usepackage{amssymb}
\usepackage{epsfig}   % Zum Einbinden von Bildern
\usepackage{url}      % Korrekter Satz von URLs
\usepackage{soulutf8}
\usepackage{color}    % Verwendung von Farben
\usepackage{listings} % Korrekter Satz von Listings und Quellcode


%Hilfs-Fonts - ohne Serifen (hier für Tabellen)
\newfont{\bib}{cmss8 scaled 1040}
\newfont{\bibf}{cmssbx8 scaled 1040}

\definecolor{lightgray}{gray}{0.85}

%Seitenformat-Definitionen
\topmargin0mm
\textwidth147mm
\textheight214mm
\evensidemargin5mm
\oddsidemargin5mm
\footskip19mm
\parindent=0in

\begin{document}          

\begin{titlepage}
  \begin{center} 
    \mbox{}
    \vspace{1cm}
    
    {\huge Titel der Seminararbeit \\[1em] {\LARGE ggf.~mit Untertitel}}  
        
    \vspace{5cm}
    
    Seminararbeit im Seminar \\[1em]
    {\large \sc Titel des Seminars} \\[1em]
    Sommersemester 20XX \\[1em]
    Hasso-Plattner-Institut für Softwaresystemtechnik GmbH \\[1em]
    Universität Potsdam
    
    \vspace{4cm}
    
		vorgelegt von
		
    \vspace{1em}
    
		{\Large Maximilian Mustermann} \\
		{\Large Alfred E. Neumann}
		
    \vspace{4em}
    
    18.~April 20XX
  \end{center}
\end{titlepage}


\setcounter{page}{1}

% Zweite Seite = Kurzzusammenfassung
\begin{center}
{\bf Kurzzusammenfassung} 
\end{center}

\noindent
An dieser Stelle erfolgt eine knappe Zusammenfassung der vorliegenden Arbeit ([engl.] Abstract)\index{Abstract}, die maximal ca.~200 Worte umfassen sollte. 
Der Sinn und Zweck dieser Kurzzusammenfassung liegt darin, einem interessierten Leser die Entscheidung zu erleichtern, die vorliegende Arbeit überhaupt zu lesen bzw.~vor dem Lesen der Arbeit erst einmal in Erfahrung zu bringen, worum es geht.
Also eine knappe, motivierende Hinführung zum Problem und wie sie es gelöst haben.

\bigskip

Wenn Sie eine Kurzzusammenfassung schreiben, bedenken Sie, dass diese oft auch alleine publiziert wird, d.h. sie sollte unabhängig vom nachfolgend explizit dargestellten Inhalt der Arbeit für den Leser verständlich sein.
Daher ist es immer sinnvoll, diese Zusammenfassung erst ganz am Ende zu schreiben, wenn Sie die eigentliche Arbeit bereits abgeschlossen haben.

\newpage

% Dritte Seite = Inhaltsverzeichnis
\tableofcontents 

\newpage

\input{chapters/chapter_1.tex} 
\newpage
%
\section{Aufbau und Inhalt der Seminararbeit}
\label{sec_aufbau}

Im vorangegangenen Kapitel hatten wir bereits die Gliederung einer Seminararbeit kurz vorgestellt und erläutert, welche inhaltlichen Punkte in der \glqq Einleitung\grqq\, behandelt werden sollten.
Die folgenden Abschnitte skizzieren inhaltlich die übrigen der bereits genannten Gliederungspunkte.

\subsection{Verwandte Arbeiten und wissenschaftlicher Hintergrund (Related Work)}
%%
Hier sind vor allem zwei inhaltliche Punkte zu berücksichtigen:
\begin{itemize}
\item {\bf Notwendige Vorarbeiten und Grundlagen, die zum Verständnis der Arbeit notwendig sind}

Keine bzw. kaum eine Arbeit beginnt als \glqq tabula rasa\grqq , d.h. meist bauen wir auf  vorhandenen Grundlagen bzw. Vorarbeiten auf.
Die zum Verständnis der eigenen Arbeit notwendigen Grundlagen und Voraussetzungen müssen in diesem Kapitel skizziert bzw. zusammengefasst werden.
Dabei sollte man vom durchschnittlichen Kenntnisstand eines Informatikers ausgehen, d.h. Allgemeinplätze und allzu Grundlegendes hat hier nichts zu suchen.
Genauso soll hier nicht notwendigerweise eine kompletter Wissenschaftszweig in epischer Tiefe ausgebreitet werden, sondern lediglich die zum Verständnis notwendigen Teilbereiche in skizzenhafter Form und mit Angabe von Literaturhinweisen zusammengefasst werden. (Zum Beispiel können hier die Grundlagen und Vorzüge von Linked Open Data erläutert werden.)

\smallskip

\item {\bf Alternative Ansätze und ggfs. Forschungsarbeiten zum Thema}
Besonders wichtig ist es, spezielle Vorarbeiten und alternative Ansätze zum behandelten Thema darzulegen.
Gibt es zu der von Ihnen gewählten Problemstellung alternative Lösungen, die einen anderen oder vergleichbaren Ansatz verfolgen? Wie unterscheiden sich diese Lösungen von Ihrem Vorschlag, wo liegen Vor- und Nachteile des jeweiligen Ansatzes?

Wichtig ist, dass Sie jede der vorgestellten Arbeiten 
\begin{itemize}
\item korrekt zitieren (Bibliografie),
\item kurz die wichtigsten Ergebnisse bzw. Strategien skizzieren und
\item diese (kurz und knapp) in Zusammenhang mit ihrer eigenen Arbeit stellen. 
\end{itemize}
Wie unterscheidet sich der eigene Ansatz von den vorgestellten Arbeiten? 
Warum ist der eigene Ansatz eventuell erfolgsversprechender? 

\end{itemize}

\subsection{Eigener Ansatz zur Lösung der gestellten Aufgabe}
%%
Hier haben Sie die Freiheit, Ihren eigenen Arbeiten angemessen viel Raum zur Verfügung zu stellen.
Achten Sie dabei auf einen logischen Aufbau der Darstellung, d.h. Grundlegendes zuerst.
\begin{itemize}
\item Wie sind Sie vorgegangen?
\item Wo gibt es Probleme?
\item Wie werden diese gelöst?
\item Schreiben Sie in verständlicher Weise und drücken Sie sich dabei jeweils möglichst präzise, d.h. unmissverständlich aus (vgl. Kap.~\ref{sec_stil})
\item Verwenden Sie Abbildungen, Tabellen und Beispiele.
\item Setzen Sie kein Wissen als implizit vorhanden voraus, sondern sprechen Sie explizit alle Probleme und wichtigen Fakten an.
\item Wichtig: Was Sie hier nicht beschreiben, können wir nicht bewerten!
\end{itemize}
Bedenken Sie dabei stets, dass ein Leser nicht dasselbe Wissen besitzen kann wie Sie und das Sie ihm deshalb ihre Ergebnisse erklären müssen.


\subsection{Diskussion der erzielten Ergebnisse}
%%
In diesem Kapitel sollten Sie Ihre Ergebnisse präsentieren.
Dabei sollten (falls jeweils zutreffend) folgende Fragen beantwortet werden:
\begin{itemize}
\item Was wurde erreicht, was kann noch verbessert werden bzw. wo gibt es noch offene (evtl. aus Zeitgründen nicht implementierte) Punkte?
\item Warum ist der eigene Ansatz besser/schlechter als die zum Vergleich herangezogenen?
\item Was haben Sie aus dem Seminar mitgenommen (z.B. Wo liegen die Vorteile von Linked Open Data?)
\end{itemize}


\subsection{Zusammenfassung und Ausblick}

In diesem Abschnitt sollten die erzielten Ergebnisse noch einmal kurz zusammengefasst werden und ein Ausblick auf weiterführende Entwicklungsarbeiten gegeben werden (vgl. Kap.~\ref{sec_conclusion}). Hier können Sie z.B. ausführen, welche Arbeiten Sie aus Zeitgründen nicht umsetzen konnten, aber für wichtig oder sinnvoll erachten.
\newpage
\input{chapters/chapter_3.tex}
\newpage
\input{chapters/chapter_4.tex}
\newpage
\input{chapters/chapter_5.tex}
\newpage
%
\section{Zusammenfassung und Ausblick}
\label{sec_conclusion}

Das Kapitel "`Zusammenfassung und Ausblick"' soll die gewonnenen Ergebnisse und Erkenntnisse Ihrer Arbeit knapp zusammenzufassen.
Stellen Sie dabei eindeutig klar, was wichtig ist und was nicht.
Dazu zählt auch, dass Sie einen Ausblick auf die Weiterentwicklung innerhalb des von Ihnen bearbeiteten Themengebiets geben können.
\begin{itemize}
\item Was haben Sie erreicht?
\item Was sind die nächsten Schritte?
\item Wie kann der vorgestellte Ansatz verwendet werden?
\end{itemize}



%Hier kommt das Literaturverzeichnis
\newpage

\addcontentsline{toc}{section}{Literaturverzeichnis} % Zeile für das Inhaltsverzeichnis

\bibliography{bibfile}
\bibliographystyle{alphadin}

\end{document}
