\newglossaryentry{UI}
{
  name=UI,
  description={Abkürzung für User Interface, die grafische Oberfläche eines Systems}
}

\newglossaryentry{breadcrumbs}
{
	name=breadcrumbs,
	description={Bei Breadcrumbs (“Brotkrümeln”) handelt es sich um einen (meist unter dem Header eines Systems) angezeigten Pfad, der dem Benutzer zeigt, über welche Schritte er an den aktuellen Ort gelangt ist}
}

\newglossaryentry{artist}
{
	name=Artist,
	despriction={Ein Artist beschreibt jede Form einer in unserem System suchbaren Entität. Dabei kann es sich um Bands, Solokünstler etc. handeln}	
}

\newglossaryentry{edge-case}
{
	name=edge-case,
	despriction={Eine Situation oder ein Problem, die/das auftritt, wenn das System mit extremen Parametern arbeitet}	
}

\newglossaryentry{uri}
{
	name=URI,
	despriction={Uniform Resource Identifier, ein String, der eine Ressource eindeutig identifizierbar macht}	
}

\newglossaryentry{template-engine}
{
	name=Template-Engine,
	despriction={eine Software, die Vorlagen-Dateien verarbeitet und Platzhalter durch aktuelle Inhalte ersetzt}	
}

\newglossaryentry{endpoint}
{
	name=Endpoint,
	despriction={eine Schnittstelle, die von Systemen nach außen Angeboten wird, um bestimmte Daten nach außen Verfügbar zu machen}	
}

\newglossaryentry{dump}
{
	name=Dump,
	despriction={ein (kompletter) Auszug aus einer Datenbank}	
}

\newglossaryentry{api}
{
	name=API,
	despriction={Application Programming Interface, eine Menge an Funktionen, die ein System nach außen anbietet}	
}

\newglossaryentry{mashup}
{
	name=Mashup,
	despriction={eine Anwendung, die mehrere verschiedene Datenquellen benutzt, um daraus einen Mehrwert zu schaffen}	
}